\chapter{Set-membership identification}    % For a new chapter (works in book and report class)

Set-membership approach is an alternative to the classical, statistical identification approach. Least-Square method is one possible example of statistical estimation algorithms in the context of estimation.\\

\subsubsection{Main ingredients}

We consider a discrete-time system described in the following parametrized regression form:\\
\begin{equation}
y(k) = f(y(k-1), y(k-2), \cdots, y(k-n), u(k), u(k-1), \cdots, u(k-m), \theta_1, \theta_2, \cdots, \theta_{n+m+1})
\end{equation}

where \(m\leq n\)\\
A-priori information on the system:
\begin{itemize}
    \item \(n\) and \(m\) are known
    \item \(f \in \mathbb{F}\) where \(\mathbb{F}\) is the class of model selected on the basis of our physical insight.
\end{itemize}


\begin{equation}
y(k) = f(y(k-1), y(k-2), \cdots, y(k-n), u(k), u(k-1), \cdots, u(k-m), \theta_1, \theta_2, \cdots, \theta_{n+m+1})
\end{equation}

A-priori information on the noise, the main difference with respect to the previous discussion.
\begin{itemize}
    \item the noise structure is known, i.e. the way the way uncertainty affects the input-output data)
    \item the noise is assumed to belong to konwn bounded set \(\mathbb{B}\).
\end{itemize}

When the consistency property of LS was discussed, and in general statistical approach to system identification, the typicall assumption on the noise is that statistical distribution of the noise sequence, or the value of some moments of inertia of the noise is known, such as variance. Here, the assumption is that the noise sequence \(\eta\) belongs to a bounded \(\mathbb{B}\). Remember that assumption two of the consistency property was the crucial one, so a change of perspective is adopted on the second assumption, where we deal with the following problem:

\textbf{Set-membership identification of LTI system} under the assumption that the uncertainty affecting the data can be modelled as an equation error \(e(k)\), which is exactly assumption \(1\) considered for the consistency theorem.\\
\[
\begin{aligned}
\tilde{y}(k) = & -\theta_1 \tilde{y}(k - 1) - \theta_2 \tilde{y}(k - 2) - \cdots - \tilde{y}(k - n) \\[1ex]
               & + \theta_{n+1} \tilde{u}(k) + \theta_{n+2} \tilde{u}(k - 1) + \cdots + \theta_{n+m+1} \tilde{u}(k - m) + \mathbf{e(k)} \\[2ex]
\end{aligned}
\]
where,
\[
\begin{aligned}
e(k) \in \mathbb{B}e = \left\{ \bar{e} = [e(1), e(2), \cdots, e(H)]^T \right\} : \left| e(k) \right| \leq \Delta e, \forall k
\end{aligned}
\]
where  \(\Delta\) is a given real bounded constant.
